\documentclass[11pt,a4paper]{article}
\usepackage[utf8]{inputenc}
\usepackage{amsmath,amssymb,amsthm}
\usepackage{physics}
\usepackage{graphicx}
\usepackage{listings}
\usepackage{xcolor}
\usepackage{geometry}
\usepackage{hyperref}
\usepackage{cleveref}

\geometry{margin=1in}

\lstset{
  language=Matlab,
  basicstyle=\ttfamily\small,
  keywordstyle=\color{blue},
  commentstyle=\color{green!60!black},
  numbers=left,
  numberstyle=\tiny,
  breaklines=true,
  frame=single
}

\newtheorem{axiom}{Axiom}
\newtheorem{definition}{Definition}
\newtheorem{theorem}{Theorem}
\newtheorem{lemma}{Lemma}
\newtheorem{corollary}{Corollary}

\title{Fabry-P\'{e}rot Optical Resonator:\\High Reflectivity Coated Mirrors\\[0.5em]\large Detailed Analysis of \texttt{Chapt9Fig6a2.m}}
\author{Generated Documentation}
\date{}

\begin{document}
\maketitle

\begin{abstract}
This document provides comprehensive theoretical foundations and line-by-line code analysis for \texttt{Chapt9Fig6a2.m}, which calculates the intensity spectrum of a Fabry-P\'{e}rot optical resonator with asymmetric high-reflectivity mirrors ($R_1 = 0.4$, $R_2 = 0.8$). The code demonstrates enhanced finesse and sharper resonance peaks compared to uncoated facets, illustrating the importance of mirror coatings in laser design.
\end{abstract}

\tableofcontents
\newpage

%==============================================================================
\section{Theoretical Foundation}
%==============================================================================

\subsection{Axioms of High-Finesse Cavities}

\begin{axiom}[Multiple Reflection Interference]
In a high-reflectivity cavity, many round trips contribute significantly to the interference pattern, leading to narrow resonances.
\end{axiom}

\begin{axiom}[Asymmetric Cavity Output]
For asymmetric mirrors ($R_1 \neq R_2$), output power is preferentially emitted from the lower-reflectivity facet.
\end{axiom}

\begin{axiom}[Photon Lifetime Enhancement]
Higher reflectivity increases the average number of round trips before photon escape:
\begin{equation}
N_{rt} \approx \frac{1}{1 - R_1 R_2}
\end{equation}
\end{axiom}

\subsection{Fundamental Definitions}

\begin{definition}[Asymmetric Finesse]
For mirrors with different reflectivities:
\begin{equation}
\mathcal{F} = \frac{\pi(R_1 R_2)^{1/4}}{1 - \sqrt{R_1 R_2}}
\end{equation}
\end{definition}

\begin{definition}[Differential Output]
The fraction of power from each mirror:
\begin{equation}
\frac{P_1}{P_{total}} = \frac{1-R_1}{(1-R_1)+(1-R_2)}, \quad \frac{P_2}{P_{total}} = \frac{1-R_2}{(1-R_1)+(1-R_2)}
\end{equation}
\end{definition}

\begin{definition}[Quality Factor]
Relation to finesse:
\begin{equation}
Q = \frac{\nu_0}{\delta\nu} = \frac{2n_r L}{\lambda} \cdot \mathcal{F} = m \cdot \mathcal{F}
\end{equation}
where $m$ is the mode number.
\end{definition}

\subsection{Core Theorems}

\begin{theorem}[Finesse Enhancement]
Increasing reflectivity from $R = 0.3$ to $R = 0.56$ (geometric mean of 0.4 and 0.8):
\begin{equation}
\frac{\mathcal{F}_{high}}{\mathcal{F}_{low}} = \frac{(1-R_{low})\sqrt{R_{high}}}{(1-R_{high})\sqrt{R_{low}}}
\end{equation}
\end{theorem}

\begin{theorem}[Cavity Loss Reduction]
The mirror loss for asymmetric cavity:
\begin{equation}
\alpha_m = \frac{1}{2L}\ln\frac{1}{R_1 R_2} = \frac{1}{2L}\ln\frac{1}{0.32} = \frac{1.14}{2L}
\end{equation}
Compare to Fresnel: $\ln(1/0.082)/2L = 2.5/2L$---reduction by factor of 2.2.
\end{theorem}

\begin{theorem}[Threshold Reduction]
The threshold gain requirement:
\begin{equation}
g_{th} = \alpha_i + \frac{1}{2L}\ln\frac{1}{R_1 R_2}
\end{equation}
Increasing $R$ directly reduces $g_{th}$ and hence $I_{th}$.
\end{theorem}

%==============================================================================
\section{Line-by-Line Code Analysis}
%==============================================================================

\subsection{Initialization and Parameters}

\begin{lstlisting}
%Chapt9Fig6a2
%Fabry-Perot optical resonator
clear;
clf;
err=0.00001;
c=3e8;                    %speed of light in vacuum
nr=3.3;                   %effective refractive index
lambda0=1310e-9;          %center emission wavelength
Lc=300e-6;                %Cavity length
\end{lstlisting}
Same cavity geometry as Fig6a1: $L = 300$ $\mu$m, $n_r = 3.3$, $\lambda_0 = 1310$ nm.

\begin{lstlisting}
r1=0.4                    %Reflectivity of mirror1
r2=0.8                    %Reflectivity of mirror2
r=r1*r2;                  %Optical loss per round-trip
\end{lstlisting}
\textbf{High-reflectivity coated mirrors:}
\begin{align}
R_1 &= 0.4 \quad \text{(output coupler)} \\
R_2 &= 0.8 \quad \text{(high reflector)} \\
R &= R_1 R_2 = 0.32
\end{align}
The asymmetric design: HR coating on rear facet, partial reflector on output facet.

\subsection{Finesse Calculation}

\begin{lstlisting}
F=pi*r^0.5/(1-r)          %Optical Finesse
Fconst=(2*F/pi)^2;
Imax=1/((1-r)^2);         %Peak intensity normalized to I0
\end{lstlisting}
\textbf{Finesse:}
\begin{equation}
\mathcal{F} = \frac{\pi\sqrt{R}}{1-R} = \frac{\pi\sqrt{0.32}}{0.68} = \frac{1.78}{0.68} = 2.61
\end{equation}
Improvement: $\mathcal{F}_{high}/\mathcal{F}_{low} = 2.61/0.98 = 2.7\times$.

\textbf{Coefficient of finesse:}
\begin{equation}
F_{const} = \left(\frac{2\mathcal{F}}{\pi}\right)^2 = \frac{4R}{(1-R)^2} = \frac{1.28}{0.462} = 2.77
\end{equation}

\textbf{Peak intensity enhancement:}
\begin{equation}
I_{max} = \frac{1}{(1-R)^2} = \frac{1}{0.462} = 2.16
\end{equation}
Enhancement doubled compared to Fresnel case.

\subsection{Spectral Parameters}

\begin{lstlisting}
f0=c*1e-12/lambda0;       %center frequency in THz
deltaf=c*1e-12/(2*Lc*nr)  %FSR in THz
deltawavelength=lambda0^2/(2*Lc*nr)  %FSR in m
\end{lstlisting}
\textbf{Free spectral range:} Same as Fig6a1 since geometry unchanged:
\begin{align}
\Delta f_{FSR} &= 0.152 \text{ THz} = 152 \text{ GHz} \\
\Delta\lambda_{FSR} &= 0.87 \text{ nm}
\end{align}

\textbf{Mode linewidth} (new, narrower):
\begin{equation}
\delta\nu = \frac{\Delta\nu_{FSR}}{\mathcal{F}} = \frac{152}{2.61} = 58 \text{ GHz}
\end{equation}

\subsection{Airy Function Computation}

\begin{lstlisting}
for i=[1:1:1000]
   
   Frequency(i)=(f0+i*0.001);
   Fwavelength(i)=c/Frequency(i);
   x=sin(pi*Frequency(i)/deltaf);
   x2=x*x;
   Intensity(i)=Imax/(1+(Fconst*(x2)));
   
end
\end{lstlisting}
\textbf{Identical algorithm} to Fig6a1, but with different $F_{const}$ and $I_{max}$:
\begin{equation}
I(\nu) = \frac{2.16}{1 + 2.77\sin^2\left(\frac{\pi\nu}{\Delta\nu_{FSR}}\right)}
\end{equation}

At anti-resonance ($\sin^2 = 1$):
\begin{equation}
I_{min} = \frac{2.16}{1 + 2.77} = 0.57
\end{equation}
Contrast ratio: $I_{max}/I_{min} = 2.16/0.57 = 3.8$.

\subsection{Plotting}

\begin{lstlisting}
figure(1);
plot(Frequency,Intensity);
axis([f0,f0+1,0,4]); 
hold on;
xlabel('Frequency, \nu(THz)'),ylabel('Intensity');
ttl=sprintf('Chapt9Fig6a2, r1=%4.2f, r2=%4.2f, nr=%4.2f, f0=%7.2e THz, Lc=%7.2e m',r1,r2,nr,f0,Lc)
title(ttl);

figure(2);
plot(Fwavelength*10^-12,Intensity);
xlabel('Wavelength, \lambda (m)'),ylabel('Intensity');
title(ttl);
\end{lstlisting}
Both frequency and wavelength domain plots with parameter annotation.

%==============================================================================
\section{Comparison: Low vs High Reflectivity}
%==============================================================================

\begin{center}
\begin{tabular}{|l|c|c|}
\hline
\textbf{Parameter} & \textbf{Fig6a1 (Fresnel)} & \textbf{Fig6a2 (Coated)} \\
\hline
$R_1$ & 0.286 & 0.40 \\
$R_2$ & 0.286 & 0.80 \\
$R = R_1 R_2$ & 0.082 & 0.32 \\
\hline
Finesse $\mathcal{F}$ & 0.98 & 2.61 \\
$F_{const}$ & 0.39 & 2.77 \\
$I_{max}$ & 1.19 & 2.16 \\
\hline
Linewidth $\delta\nu$ & 155 GHz & 58 GHz \\
Q factor ($m=1500$) & 1,470 & 3,915 \\
\hline
Mirror loss $\alpha_m$ & 83 cm$^{-1}$ & 38 cm$^{-1}$ \\
\hline
\end{tabular}
\end{center}

%==============================================================================
\section{Physical Interpretation}
%==============================================================================

\subsection{Improved Mode Selectivity}

With higher finesse:
\begin{itemize}
\item Resonance peaks are sharper ($\delta\nu$ reduced by 2.7$\times$)
\item Better discrimination between adjacent modes
\item Improved single-mode operation potential
\end{itemize}

\subsection{Threshold Reduction}

The mirror loss reduction from 83 to 38 cm$^{-1}$ means:
\begin{equation}
\Delta g_{th} = 45 \text{ cm}^{-1}
\end{equation}
For typical gain slope $g_0 \sim 2000$ cm$^{-1}$/(10$^{18}$ cm$^{-3}$), this saves:
\begin{equation}
\Delta n_{th} = \frac{45}{2000} \times 10^{18} = 2.25 \times 10^{16} \text{ cm}^{-3}
\end{equation}

\subsection{Asymmetric Design Rationale}

The choice $R_1 = 0.4 < R_2 = 0.8$:
\begin{itemize}
\item Most power exits through facet 1 (lower $R$)
\item Rear facet (high $R$) acts as ``back reflector''
\item Photon makes more round trips before escaping
\item Common in single-sided output devices
\end{itemize}

Power distribution:
\begin{equation}
\frac{P_1}{P_2} = \frac{1-R_1}{1-R_2} = \frac{0.6}{0.2} = 3:1
\end{equation}
75\% of output from facet 1.

\subsection{Still Moderate Finesse}

Even with coatings, $\mathcal{F} = 2.6$ is still relatively low:
\begin{itemize}
\item Mode linewidth 58 GHz still significant
\item Adjacent modes at 152 GHz spacing show overlap
\item For true single-mode operation, need $\mathcal{F} > 10$ or additional mode selection
\end{itemize}

%==============================================================================
\section{Numerical Methods}
%==============================================================================

\subsection{Resolution Adequacy}

With 1 GHz sampling and 58 GHz linewidth:
\begin{equation}
\frac{\delta\nu}{\Delta\nu_{sample}} = 58
\end{equation}
Still adequate resolution (58 points across FWHM).

\subsection{Contrast Visibility}

The visibility of the interference pattern:
\begin{equation}
V = \frac{I_{max} - I_{min}}{I_{max} + I_{min}} = \frac{2.16 - 0.57}{2.16 + 0.57} = 0.58
\end{equation}
Improved from $V = 0.09$ in the Fresnel case.

%==============================================================================
\section{Summary}
%==============================================================================

This code simulates a Fabry-P\'{e}rot resonator with asymmetric coated mirrors:
\begin{itemize}
\item Output coupler: $R_1 = 0.4$
\item High reflector: $R_2 = 0.8$
\item Same 300 $\mu$m cavity as Fig6a1
\end{itemize}

Key improvements over uncoated cavity:
\begin{itemize}
\item Finesse increased 2.7$\times$ (0.98 $\to$ 2.61)
\item Linewidth reduced 2.7$\times$ (155 $\to$ 58 GHz)
\item Peak enhancement nearly doubled (1.19 $\to$ 2.16)
\item Mirror loss reduced 2.2$\times$ (83 $\to$ 38 cm$^{-1}$)
\end{itemize}

This represents a practical laser design with preferential output from one facet, though even higher reflectivities would further improve performance.

\end{document}
