\documentclass[11pt,a4paper]{article}
\usepackage[utf8]{inputenc}
\usepackage{amsmath,amssymb,amsthm}
\usepackage{physics}
\usepackage{listings}
\usepackage{xcolor}
\usepackage{geometry}
\usepackage{hyperref}
\usepackage{booktabs}
\usepackage{graphicx}

\geometry{margin=1in}

\lstset{
    language=Matlab,
    basicstyle=\ttfamily\small,
    keywordstyle=\color{blue},
    commentstyle=\color{green!60!black},
    stringstyle=\color{orange},
    numbers=left,
    numberstyle=\tiny\color{gray},
    breaklines=true,
    frame=single,
    backgroundcolor=\color{gray!10}
}

\newtheorem{axiom}{Axiom}
\newtheorem{definition}{Definition}
\newtheorem{theorem}{Theorem}
\newtheorem{lemma}{Lemma}
\newtheorem{corollary}{Corollary}

\title{\textbf{Chapt7Fig6b.m} \\ Fermi-Dirac Distribution at Lower Carrier Density \\ \large Non-Degenerate Semiconductor Regime}
\author{Semiconductor Physics Documentation}
\date{}

\begin{document}

\maketitle

\begin{abstract}
This document analyzes \texttt{Chapt7Fig6b.m}, which plots the Fermi-Dirac distribution function at a carrier density of $n = 10^{17}$ cm$^{-3}$, one order of magnitude lower than \texttt{Chapt7Fig6a.m}. This lower density places the system closer to the non-degenerate (Maxwell-Boltzmann) regime, providing insight into the classical-quantum transition.
\end{abstract}

\tableofcontents
\newpage

%==============================================================================
\section{Theoretical Foundation}
%==============================================================================

\subsection{Fermi-Dirac Statistics Review}

\begin{theorem}[Fermi-Dirac Distribution]
The probability of occupation of a quantum state with energy $E$ is:
\begin{equation}
f(E) = \frac{1}{e^{(E-\mu)/k_B T} + 1}
\end{equation}
where $\mu$ is the chemical potential (Fermi level) and $T$ is the temperature.
\end{theorem}

\subsection{Degeneracy Parameter}

\begin{definition}[Degeneracy Parameter]
The degeneracy of an electron gas is characterized by:
\begin{equation}
\eta = \frac{\mu}{k_B T}
\end{equation}
\begin{itemize}
\item $\eta \gg 1$: Degenerate regime (quantum statistics essential)
\item $\eta \ll -1$: Non-degenerate regime (classical approximation valid)
\item $|\eta| \lesssim 1$: Transition regime
\end{itemize}
\end{definition}

\subsection{Non-Degenerate Limit: Maxwell-Boltzmann Approximation}

\begin{theorem}[Maxwell-Boltzmann Limit]
When $E - \mu \gg k_B T$ for all occupied states, the Fermi-Dirac distribution reduces to:
\begin{equation}
f(E) \approx e^{-(E-\mu)/k_B T}
\end{equation}
\end{theorem}

\begin{proof}
When $(E-\mu)/k_B T \gg 1$, we have $e^{(E-\mu)/k_B T} \gg 1$, so:
\begin{equation}
f(E) = \frac{1}{e^{(E-\mu)/k_B T} + 1} \approx \frac{1}{e^{(E-\mu)/k_B T}} = e^{-(E-\mu)/k_B T}
\end{equation}
\end{proof}

\subsection{Chemical Potential Dependence on Carrier Density}

\begin{theorem}[Non-Degenerate Carrier Density]
In the non-degenerate limit, the carrier density is:
\begin{equation}
n = N_c \exp\left(\frac{\mu}{k_B T}\right)
\end{equation}
where $N_c = 2\left(\frac{m^* k_B T}{2\pi\hbar^2}\right)^{3/2}$ is the effective density of states.
\end{theorem}

\begin{corollary}[Chemical Potential in Non-Degenerate Regime]
\begin{equation}
\mu = k_B T \ln\left(\frac{n}{N_c}\right)
\end{equation}
When $n < N_c$, the chemical potential is negative (below the band edge).
\end{corollary}

\subsection{Effective Density of States for GaAs}

For GaAs with $m^* = 0.07 m_0$:
\begin{equation}
N_c = 2\left(\frac{0.07 m_0 k_B T}{2\pi\hbar^2}\right)^{3/2}
\end{equation}

At $T = 300$ K:
\begin{equation}
N_c \approx 4.7 \times 10^{17} \text{ cm}^{-3}
\end{equation}

Since $n = 10^{17}$ cm$^{-3} < N_c$, the system is near the non-degenerate regime at room temperature.

%==============================================================================
\section{Line-by-Line Code Analysis}
%==============================================================================

\subsection{Key Parameter Difference}

\begin{lstlisting}
n=1.e17
\end{lstlisting}
\begin{equation}
n = 1 \times 10^{17} \text{ cm}^{-3}
\end{equation}
\textit{This is the critical difference from Chapt7Fig6a.m. The carrier density is 10 times lower, pushing the system toward the non-degenerate regime.}

\subsection{Physical Constants (Identical to Chapt7Fig6a)}

\begin{lstlisting}
m0=9.109382;
\end{lstlisting}
\begin{equation}
m_0 = 9.109382 \times 10^{-31} \text{ kg}
\end{equation}
\textit{Bare electron mass.}

\vspace{0.5em}
\begin{lstlisting}
m1=0.07;
\end{lstlisting}
\begin{equation}
m^* = 0.07 \, m_0
\end{equation}
\textit{Effective mass for GaAs conduction band electrons.}

\vspace{0.5em}
\begin{lstlisting}
echarge=1.6021764;
\end{lstlisting}
\begin{equation}
e = 1.6021764 \times 10^{-19} \text{ C}
\end{equation}
\textit{Elementary charge.}

\vspace{0.5em}
\begin{lstlisting}
hbar=1.05457159;
\end{lstlisting}
\begin{equation}
\hbar = 1.05457159 \times 10^{-34} \text{ J}\cdot\text{s}
\end{equation}
\textit{Reduced Planck constant.}

\vspace{0.5em}
\begin{lstlisting}
kB=8.61734e-5;
\end{lstlisting}
\begin{equation}
k_B = 8.61734 \times 10^{-5} \text{ eV/K}
\end{equation}
\textit{Boltzmann constant in convenient units.}

\vspace{0.5em}
\begin{lstlisting}
rerr=1.e-3;
\end{lstlisting}
\begin{equation}
\epsilon_{\text{rel}} = 10^{-3}
\end{equation}
\textit{Relative error tolerance for chemical potential calculation.}

\subsection{Temperature Array}

\begin{lstlisting}
Temperature=[0.1,100,200,300,400];
\end{lstlisting}
\begin{equation}
T \in \{0.1, 100, 200, 300, 400\} \text{ K}
\end{equation}
\textit{Same temperature range as Chapt7Fig6a for direct comparison.}

\vspace{0.5em}
\begin{lstlisting}
nTemp=length(Temperature)
\end{lstlisting}
\begin{equation}
N_{\text{temp}} = 5
\end{equation}
\textit{Number of temperature curves to plot.}

\subsection{Energy Grid Setup}

\begin{lstlisting}
nplotpoints=300;
Emin=0;
Emax=200;
Estep=(Emax-Emin)/nplotpoints;
\end{lstlisting}
\begin{equation}
E_k = k \cdot \frac{200 \text{ meV}}{300} \quad \text{for } k = 0, 1, \ldots, 300
\end{equation}
\textit{Energy discretization from 0 to 200 meV.}

\subsection{Main Computation Loop}

\begin{lstlisting}
for j=1:1:nTemp
   kelvin=Temperature(j);
   kBT=1000.*kelvin*kB;
   beta=1./kBT;
\end{lstlisting}
\begin{equation}
\beta = \frac{1}{k_B T} \quad [\text{meV}^{-1}]
\end{equation}
\textit{Compute inverse thermal energy for each temperature.}

\vspace{0.5em}
\begin{lstlisting}
   mu1=chempot(kelvin,m1,n,rerr);
\end{lstlisting}
\begin{equation}
\mu = \mu(T, m^* = 0.07m_0, n = 10^{17} \text{ cm}^{-3})
\end{equation}
\textit{At lower carrier density, $\mu$ will be smaller (possibly negative at high $T$).}

\subsection{Fermi Function Evaluation Loop}

\begin{lstlisting}
  for k = 1:1:nplotpoints+1
     energy=((k-1)*Estep)+Emin;
     E(k)=energy;
     Prob=fermi(beta,energy,mu1);
     FD(k)=Prob;
  end;
\end{lstlisting}
\begin{equation}
f(E_k) = \frac{1}{\exp[\beta(E_k - \mu)] + 1}
\end{equation}
\textit{Evaluate Fermi-Dirac distribution at each energy point.}

\subsection{Plotting}

\begin{lstlisting}
    hold on;figure(1);plot(E,log(FD));axis([0 Emax -10 0]);
end
xlabel('Energy, E (meV)');
ylabel('ln(occupation factor)');
ttl=sprintf('Chapt7Fig6b, m* = %4.2f m0, Tmax = %4.1f K, n = %7.2e cm-3',m1,kelvin,n);
title(ttl);
\end{lstlisting}
\begin{equation}
\ln f(E) \text{ vs } E
\end{equation}
\textit{Logarithmic plot showing the exponential tail behavior more prominently.}

%==============================================================================
\section{Physical Interpretation and Comparison}
%==============================================================================

\subsection{Effect of Lower Carrier Density}

Comparing $n = 10^{17}$ cm$^{-3}$ (this code) with $n = 10^{18}$ cm$^{-3}$ (Chapt7Fig6a):

\begin{table}[h]
\centering
\begin{tabular}{lcc}
\toprule
Property & $n = 10^{17}$ cm$^{-3}$ & $n = 10^{18}$ cm$^{-3}$ \\
\midrule
Chemical potential $\mu$ & Lower & Higher \\
Degeneracy $\eta = \mu/k_B T$ & Smaller & Larger \\
Regime & Non-degenerate & Degenerate \\
\bottomrule
\end{tabular}
\caption{Comparison of regimes}
\end{table}

\subsection{Chemical Potential Estimates}

At $T = 300$ K with $N_c \approx 4.7 \times 10^{17}$ cm$^{-3}$:

For $n = 10^{17}$ cm$^{-3}$:
\begin{equation}
\mu \approx k_B T \ln\left(\frac{10^{17}}{4.7 \times 10^{17}}\right) \approx k_B T \ln(0.21) \approx -1.56 \, k_B T \approx -40 \text{ meV}
\end{equation}

For $n = 10^{18}$ cm$^{-3}$:
\begin{equation}
\mu \approx k_B T \ln\left(\frac{10^{18}}{4.7 \times 10^{17}}\right) \approx k_B T \ln(2.13) \approx 0.76 \, k_B T \approx 20 \text{ meV}
\end{equation}

\subsection{Implications for the Plot}

With lower carrier density:
\begin{itemize}
\item The chemical potential is lower (possibly negative at high $T$)
\item The Fermi function drops off more steeply at low energies
\item The system more closely follows Maxwell-Boltzmann statistics
\item The logarithmic plot shows nearly linear behavior over a wider range
\end{itemize}

\subsection{Temperature Dependence}

At very low temperature ($T = 0.1$ K):
\begin{equation}
k_B T \approx 8.6 \times 10^{-6} \text{ eV} = 8.6 \times 10^{-3} \text{ meV}
\end{equation}

Even at low carrier density, the thermal energy is so small that the distribution still shows sharp step-like behavior, with all electrons in states below the Fermi energy.

At $T = 400$ K:
\begin{equation}
k_B T \approx 34.5 \text{ meV}
\end{equation}

The distribution is significantly broadened, and with $\mu < 0$, the occupation probability at $E = 0$ is already less than 0.5.

%==============================================================================
\section{Numerical Considerations}
%==============================================================================

\subsection{The chempot Function}

The function \texttt{chempot(kelvin, m1, n, rerr)} solves:
\begin{equation}
n = \frac{1}{2\pi^2}\left(\frac{2m^*}{\hbar^2}\right)^{3/2} \int_0^\infty \frac{\sqrt{E}}{e^{(E-\mu)/k_B T} + 1} dE
\end{equation}

This is typically done using:
\begin{enumerate}
\item Initial guess based on non-degenerate approximation: $\mu_0 = k_B T \ln(n/N_c)$
\item Newton-Raphson iteration to refine
\item Convergence criterion: $|\mu_{i+1} - \mu_i| < \epsilon_{\text{rel}} \cdot |\mu_i|$
\end{enumerate}

\subsection{The fermi Function}

The function \texttt{fermi(beta, E, mu)} computes:
\begin{equation}
f = \frac{1}{e^{\beta(E-\mu)} + 1}
\end{equation}

Numerical care is needed for large arguments:
\begin{itemize}
\item If $\beta(E-\mu) > 700$: $f \approx 0$
\item If $\beta(E-\mu) < -700$: $f \approx 1$
\end{itemize}

%==============================================================================
\section{Summary}
%==============================================================================

This code demonstrates the Fermi-Dirac distribution in the non-degenerate regime, where the carrier density is comparable to or less than the effective density of states. The key physical insight is that reducing the carrier density by a factor of 10 shifts the chemical potential significantly lower, bringing the system closer to classical (Maxwell-Boltzmann) behavior while still retaining quantum effects at low temperatures.

\end{document}
