\documentclass[11pt,a4paper]{article}
\usepackage[utf8]{inputenc}
\usepackage{amsmath,amssymb,amsthm}
\usepackage{physics}
\usepackage{listings}
\usepackage{xcolor}
\usepackage{geometry}
\usepackage{hyperref}
\usepackage{booktabs}

\geometry{margin=1in}

\lstset{
    language=Matlab,
    basicstyle=\ttfamily\small,
    keywordstyle=\color{blue},
    commentstyle=\color{green!60!black},
    stringstyle=\color{orange},
    numbers=left,
    numberstyle=\tiny\color{gray},
    breaklines=true,
    frame=single,
    backgroundcolor=\color{gray!10}
}

\newtheorem{axiom}{Axiom}
\newtheorem{definition}{Definition}
\newtheorem{theorem}{Theorem}
\newtheorem{lemma}{Lemma}
\newtheorem{corollary}{Corollary}

\title{\textbf{Chapt8Exercise7b.m} \\ Thomas-Fermi vs RPA Screening at Low Density \\ \large Breakdown of Thomas-Fermi Approximation}
\author{Semiconductor Physics Documentation}
\date{}

\begin{document}

\maketitle

\begin{abstract}
This document analyzes \texttt{Chapt8Exercise7b.m}, which compares Thomas-Fermi and RPA scattering rates at a low carrier density of $n = 10^{14}$ cm$^{-3}$. At this density, the Thomas-Fermi approximation breaks down, revealing the importance of non-local screening effects captured by RPA.
\end{abstract}

\tableofcontents
\newpage

%==============================================================================
\section{Theoretical Foundation}
%==============================================================================

\subsection{Validity Conditions for Thomas-Fermi Screening}

\begin{theorem}[Thomas-Fermi Validity]
The Thomas-Fermi approximation is valid when:
\begin{equation}
q \ll k_F \quad \text{and} \quad q_{\text{TF}} \sim k_F
\end{equation}
This requires the electron gas to be sufficiently dense (degenerate).
\end{theorem}

\begin{definition}[Degeneracy Criterion]
An electron gas is degenerate when:
\begin{equation}
E_F \gg k_B T
\end{equation}
where the Fermi energy is $E_F = \hbar^2 k_F^2/(2m^*)$.
\end{definition}

\subsection{Low-Density Regime}

At $n = 10^{14}$ cm$^{-3}$:
\begin{equation}
k_F = (3\pi^2 n)^{1/3} = (3\pi^2 \times 10^{20})^{1/3} \approx 1.43 \times 10^{7} \text{ m}^{-1}
\end{equation}

Compare to $n = 10^{18}$ cm$^{-3}$:
\begin{equation}
k_F = 3.09 \times 10^{8} \text{ m}^{-1}
\end{equation}

The Fermi wavevector decreases by a factor of $\sim 20$.

\subsection{Consequences for Screening}

\begin{theorem}[Weak Screening Limit]
At low density:
\begin{equation}
q_{\text{TF}} \propto \sqrt{k_F} \propto n^{1/6}
\end{equation}
The screening length increases, making screening less effective.
\end{theorem}

\begin{corollary}[RPA Corrections]
When $q \gtrsim k_F$, the RPA dielectric function deviates significantly from Thomas-Fermi:
\begin{equation}
\varepsilon_{\text{RPA}}(q) \neq \varepsilon_{\text{TF}}(q) \quad \text{for } q \gtrsim k_F
\end{equation}
\end{corollary}

%==============================================================================
\section{Line-by-Line Code Analysis}
%==============================================================================

\subsection{Key Parameter: Low Carrier Density}

\begin{lstlisting}
clear;clf;
n=1e14;
n=n*1e6;
\end{lstlisting}
\begin{equation}
n = 10^{14} \text{ cm}^{-3} = 10^{20} \text{ m}^{-3}
\end{equation}
\textit{Critical difference: carrier density is $10^4$ times lower than Chapt8Exercise7a.}

\subsection{Physical Constants (Same as 7a)}

\begin{lstlisting}
m0=9.109382e-31;
echarge=1.6021764e-19;
epsilon0=8.85419e-12;
hbar=1.05457159e-34;
hbar3=hbar^3;
m=0.07*m0;
epsilonr0=13.2;
epsilon=epsilon0*epsilonr0;
\end{lstlisting}
\textit{Material parameters for GaAs remain unchanged.}

\subsection{Fermi Wavevector}

\begin{lstlisting}
kF=(3*(pi^2)*n)^(1/3)
\end{lstlisting}
\begin{equation}
k_F = (3\pi^2 \times 10^{20})^{1/3} \approx 1.43 \times 10^{7} \text{ m}^{-1}
\end{equation}
\textit{Much smaller than at high density, indicating weak degeneracy.}

\subsection{Energy Loop}

\begin{lstlisting}
E=-0.1*echarge;
for j=1:1:2
E=E+0.2*echarge;
   k=sqrt(2*m*E)/hbar;
k3=k^3;
\end{lstlisting}
\begin{equation}
E \in \{0.1, 0.3\} \text{ eV}, \quad k = \frac{\sqrt{2m^* E}}{\hbar}
\end{equation}
\textit{Same electron energies as Chapt8Exercise7a for comparison.}

\subsection{RPA Constants}

\begin{lstlisting}
   rs0=((3/(4*pi*n))^(1/3))*(m*(echarge^2)/(4*pi*epsilon0*(hbar^2)));
   xi=((32*(pi^2)/9)^(1/3))/(pi^2);
\end{lstlisting}
\begin{equation}
r_s = \left(\frac{3}{4\pi n}\right)^{1/3} \cdot \frac{m^* e^2}{4\pi\varepsilon_0\hbar^2}
\end{equation}
\textit{At low density, $r_s$ is larger, indicating stronger electron-electron correlations.}

\subsection{Thomas-Fermi Wavevector}

\begin{lstlisting}
  qTF=sqrt(kF*m*echarge^2/(epsilon*(pi^2)*(hbar^2)));
\end{lstlisting}
\begin{equation}
q_{\text{TF}} = \sqrt{\frac{k_F m^* e^2}{\varepsilon \pi^2 \hbar^2}}
\end{equation}
\textit{Smaller $k_F$ leads to smaller $q_{\text{TF}}$, weaker screening.}

\subsection{Angular Discretization}

\begin{lstlisting}
theta=[pi/1800:pi/1800:pi];
	q=2*k*sin(theta/2);
   eta=sin(theta/2);
   deta=pi*cos(theta/2)./2/180;
   eta3=eta.^3;
\end{lstlisting}
\textit{Same angular grid as Chapt8Exercise7a.}

\subsection{RPA Dielectric Function}

\begin{lstlisting}
   x=q/kF;
   absx=abs(x+2)./abs(x-2);
   RPAepsilonr=epsilonr0+((rs0./x.^3)*xi.*(x+((1-((x.^2)/4)).*log(absx))));
\end{lstlisting}
\begin{equation}
x = q/k_F \quad \text{(now larger values since $k_F$ is smaller)}
\end{equation}
\textit{At low density, $x = q/k_F$ can be $\gg 1$, where TF fails.}

\subsection{Thomas-Fermi Dielectric Function}

\begin{lstlisting}
   TFepsilon=epsilon*(1+qTF^2./q.^2);
\end{lstlisting}
\begin{equation}
\varepsilon_{\text{TF}} = \varepsilon\left(1 + \frac{q_{\text{TF}}^2}{q^2}\right)
\end{equation}
\textit{With smaller $q_{\text{TF}}$, the screening term is less effective.}

\subsection{Scattering Rates}

\begin{lstlisting}
   RPArate=2*pi*m/hbar3/k3*n*(echarge^2/4/pi/epsilon0)^2.*deta./RPAepsilonr.^2./eta3;
   TFrate =2*pi*m/hbar3/k3*n*(echarge^2/4/pi)^2.*deta./TFepsilon.^2./eta3;
\end{lstlisting}
\begin{equation}
\frac{1}{\tau} = \frac{2\pi m^*}{\hbar^3 k^3} n \left(\frac{e^2}{4\pi}\right)^2 \frac{d\eta}{\varepsilon(q)^2 \eta^3}
\end{equation}
\textit{Same formula, but $n$ is lower and screening is weaker.}

\subsection{Plotting with Restricted Axis}

\begin{lstlisting}
  plot(theta*180/pi, TFrate,'r');
  axis([0,35,0,9e9]);
  ...
  plot(theta*180/pi, RPArate,'b');
\end{lstlisting}
\textit{Axis restricted to $\theta \in [0°, 35°]$ to focus on the forward-scattering region where differences are largest.}

%==============================================================================
\section{Physical Interpretation}
%==============================================================================

\subsection{Why TF Fails at Low Density}

The Thomas-Fermi approximation assumes:
\begin{enumerate}
\item Local response: Perturbation at $\mathbf{r}$ only affects density at $\mathbf{r}$
\item Degenerate limit: All states up to $E_F$ are filled
\item Slow spatial variation: $q \ll k_F$
\end{enumerate}

At $n = 10^{14}$ cm$^{-3}$:
\begin{itemize}
\item Fermi energy: $E_F \approx 0.1$ meV (non-degenerate at room temperature)
\item Typical scattering wavevector: $q \sim k \gg k_F$
\item TF assumption $q \ll k_F$ is violated
\end{itemize}

\subsection{RPA Captures the Correct Physics}

The RPA (Lindhard) function includes:
\begin{equation}
\varepsilon(q) = 1 + \frac{q_{\text{TF}}^2}{q^2} \cdot g(q/k_F)
\end{equation}

where $g(x)$ is a correction factor that:
\begin{itemize}
\item Equals 1 for $x \ll 1$ (recovers TF)
\item Decreases for $x \gtrsim 1$ (weakens screening)
\item Vanishes for $x \gg 1$ (bare Coulomb restored)
\end{itemize}

\subsection{Expected Results}

\begin{itemize}
\item TF (red curve): Underestimates scattering at large angles
\item RPA (blue curve): Shows stronger scattering due to weaker screening
\item Difference is most pronounced at large $q$ (large angles)
\end{itemize}

%==============================================================================
\section{Comparison with Chapt8Exercise7a}
%==============================================================================

\begin{table}[h]
\centering
\begin{tabular}{lcc}
\toprule
Parameter & Chapt8Exercise7a & Chapt8Exercise7b \\
\midrule
Carrier density $n$ & $10^{18}$ cm$^{-3}$ & $10^{14}$ cm$^{-3}$ \\
Fermi wavevector $k_F$ & $3.09 \times 10^8$ m$^{-1}$ & $1.43 \times 10^7$ m$^{-1}$ \\
Screening & Strong & Weak \\
TF validity & Good & Poor \\
TF-RPA agreement & Good & Poor \\
\bottomrule
\end{tabular}
\caption{Comparison of the two exercises}
\end{table}

%==============================================================================
\section{Summary}
%==============================================================================

This code demonstrates the breakdown of the Thomas-Fermi approximation at low carrier densities. When the electron gas is non-degenerate and the typical scattering wavevector exceeds the Fermi wavevector, the local Thomas-Fermi approximation fails, and the full RPA treatment becomes necessary. The restricted angular range in the plot highlights the region where differences between TF and RPA are most pronounced.

\end{document}
