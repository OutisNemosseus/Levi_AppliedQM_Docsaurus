\documentclass[11pt,a4paper]{article}
\usepackage[utf8]{inputenc}
\usepackage{amsmath,amssymb,amsthm}
\usepackage{physics}
\usepackage{listings}
\usepackage{xcolor}
\usepackage{geometry}
\usepackage{hyperref}
\usepackage{booktabs}

\geometry{margin=1in}

\lstset{
    language=Matlab,
    basicstyle=\ttfamily\small,
    keywordstyle=\color{blue},
    commentstyle=\color{green!60!black},
    stringstyle=\color{orange},
    numbers=left,
    numberstyle=\tiny\color{gray},
    breaklines=true,
    frame=single,
    backgroundcolor=\color{gray!10}
}

\newtheorem{axiom}{Axiom}
\newtheorem{definition}{Definition}
\newtheorem{theorem}{Theorem}
\newtheorem{lemma}{Lemma}
\newtheorem{corollary}{Corollary}

\title{\textbf{Chapt9Exercise2.m} \\ Peak Gain and Total Spontaneous Emission \\ \large Carrier Density Dependence for Laser Design}
\author{Semiconductor Physics Documentation}
\date{}

\begin{document}

\maketitle

\begin{abstract}
This document analyzes \texttt{Chapt9Exercise2.m}, which calculates the peak optical gain and total spontaneous emission as functions of carrier density. These relationships are fundamental for semiconductor laser design, determining threshold current and efficiency.
\end{abstract}

\tableofcontents
\newpage

%==============================================================================
\section{Theoretical Foundation}
%==============================================================================

\subsection{Gain-Carrier Density Relationship}

\begin{theorem}[Empirical Gain Model]
The peak gain can be approximated by a logarithmic or linear relationship:
\begin{equation}
g_{\max} \approx g_0 \ln\left(\frac{n}{n_0}\right) \quad \text{or} \quad g_{\max} \approx a(n - n_0)
\end{equation}
where $n_0$ is the transparency carrier density and $g_0$ (or $a$) is the gain coefficient.
\end{theorem}

\begin{definition}[Transparency Carrier Density]
At transparency ($g = 0$), the quasi-Fermi level separation equals the bandgap:
\begin{equation}
\Delta\mu = \mu_e + \mu_h = E_g
\end{equation}
(in the notation where energies are measured from band edges)
\end{definition}

\subsection{Total Spontaneous Emission}

\begin{definition}[Integrated Spontaneous Emission]
The total spontaneous emission rate is:
\begin{equation}
R_{sp} = \int_0^\infty r_{sp}(E) \, dE = \int_0^\infty C\sqrt{E} \cdot f_e(E) \cdot f_h(E) \, dE
\end{equation}
\end{definition}

\begin{theorem}[Carrier Lifetime]
The radiative recombination rate defines the spontaneous lifetime:
\begin{equation}
R_{sp} = \frac{n}{\tau_{sp}} = Bn^2
\end{equation}
where $B$ is the radiative recombination coefficient (for non-degenerate case).
\end{theorem}

\subsection{Quasi-Fermi Level Separation}

\begin{theorem}[Transparency Condition]
For population inversion to exist at the bandgap energy:
\begin{equation}
\mu_e + \mu_h > E_g \quad \Rightarrow \quad \Delta\mu > 0
\end{equation}
(measuring chemical potentials from band edges)
\end{theorem}

%==============================================================================
\section{Line-by-Line Code Analysis}
%==============================================================================

\subsection{Physical Constants and Material Parameters}

\begin{lstlisting}
clf
   echarge=1.6021764e-19;
   hbar=1.05457159e-34;
   c = 2.99792458e8;
   kB=8.61734e-5;
   epsilon0=8.8541878e-12;
\end{lstlisting}
\textit{Fundamental constants.}

\vspace{0.5em}
\begin{lstlisting}
   m0=9.109382e-31;
   me=0.07*m0;
   mhh=0.5*m0;
   mr=1/(1/me+1/mhh);
   rerr=1e-3;
\end{lstlisting}
\begin{equation}
m_e = 0.07 m_0, \quad m_{hh} = 0.5 m_0, \quad m_r = \frac{m_e m_{hh}}{m_e + m_{hh}}
\end{equation}
\textit{GaAs effective masses.}

\vspace{0.5em}
\begin{lstlisting}
   nr=3.3;
   Eg=1.4
\end{lstlisting}
\begin{equation}
n_r = 3.3, \quad E_g = 1.4 \text{ eV}
\end{equation}
\textit{GaAs optical and electronic properties.}

\subsection{Carrier Density Range}

\begin{lstlisting}
   n1=1e18;
   ncarrier1=n1*1e6;
   n2=1e19;
   ncarrier2=n2*1e6;
\end{lstlisting}
\begin{equation}
n \in [10^{18}, 10^{19}] \text{ cm}^{-3}
\end{equation}
\textit{Carrier density range spanning typical laser operation.}

\subsection{Temperature}

\begin{lstlisting}
   kelvin=300.0;
   kBT=kB*kelvin;
   beta=1/kBT;
\end{lstlisting}
\begin{equation}
T = 300 \text{ K}, \quad k_B T \approx 25.9 \text{ meV}
\end{equation}
\textit{Room temperature operation.}

\subsection{Gain Constant}

\begin{lstlisting}
   const=2.64e4;
\end{lstlisting}
\begin{equation}
C = 2.64 \times 10^4 \text{ cm}^{-1}
\end{equation}
\textit{Calibrated to give realistic gain values.}

\subsection{Main Loop Over Carrier Density}

\begin{lstlisting}
   deltacarrier=(ncarrier2-ncarrier1)/100;
   for k=1:100
      ncarrier(k)=ncarrier1+(k-1)*deltacarrier;
\end{lstlisting}
\begin{equation}
n_k = 10^{18} + (k-1) \times 9 \times 10^{22} \text{ m}^{-3}
\end{equation}
\textit{100 carrier density points from $10^{18}$ to $10^{19}$ cm$^{-3}$.}

\subsection{Chemical Potential Calculation}

\begin{lstlisting}
      muhh(k)=mu(mhh,ncarrier(k),kelvin,rerr);
   mue(k) =mu(me, ncarrier(k),kelvin,rerr);
   deltamu(k)=mue(k)+muhh(k);
\end{lstlisting}
\begin{equation}
\mu_h(n_k), \quad \mu_e(n_k), \quad \Delta\mu(n_k) = \mu_e + \mu_h
\end{equation}
\textit{Calculate quasi-Fermi levels and their sum for each carrier density.}

\subsection{Energy Integration Loop}

\begin{lstlisting}
      deltae=0.001;
      rspon=0;
      peakgain(k)=-10000;
      rspontotal(k)=0;
\end{lstlisting}
\textit{Initialize with $\Delta E = 1$ meV, large negative initial peak gain.}

\vspace{0.5em}
\begin{lstlisting}
   	for j=1:500
      	Energy(j)=j*deltae;
      	Ehh=Energy(j)/(1+mhh/me);
      	Ee=Energy(j)/(1+me/mhh);
      	fhh=fermi(beta,Ehh,muhh(k));
      	fe=fermi(beta,Ee,mue(k));
\end{lstlisting}
\begin{equation}
E_j \in [1, 500] \text{ meV}, \quad E_h, E_e, f_h, f_e
\end{equation}
\textit{Loop over energies and compute Fermi functions.}

\subsection{Gain and Spontaneous Emission Computation}

\begin{lstlisting}
         gain(j)=const*(Energy(j)^0.5)*(fhh+fe-1);
         rspon=rspon+(const*(Energy(j)^0.5))*(fe*fhh)*deltae;
         if(gain(j) > peakgain(k))
            peakgain(k)=gain(j);
         end
\end{lstlisting}
\begin{equation}
g(E) = C\sqrt{E}(f_e + f_h - 1)
\end{equation}
\begin{equation}
R_{sp} = \sum_j C\sqrt{E_j} \cdot f_e \cdot f_h \cdot \Delta E
\end{equation}
\textit{Track maximum gain and integrate spontaneous emission.}

\vspace{0.5em}
\begin{lstlisting}
      rspontotal(k)=rspon;
\end{lstlisting}
\textit{Store total spontaneous emission for this carrier density.}

\subsection{Plotting Results}

\begin{lstlisting}
   figure(1)
   plot(ncarrier./1e6, deltamu);
   xlabel('Carrier concentration, n (cm^{-3})');
   ylabel('Difference in chemical potential,  \Delta\mu (eV)');
\end{lstlisting}
\textit{Plot 1: $\Delta\mu$ vs carrier density.}

\vspace{0.5em}
\begin{lstlisting}
   figure(2)
   plot(ncarrier./1e6, peakgain);
   xlabel('Carrier concentration, n (cm^{-3})');
   ylabel('Peak optical gain, g (cm^{-1})');
\end{lstlisting}
\textit{Plot 2: Peak gain vs carrier density.}

\vspace{0.5em}
\begin{lstlisting}
   figure(3)
   plot(ncarrier./1e6, rspontotal);
   xlabel('Carrier concentration, n (cm^{-3})');
   ylabel('Total spontaneous emission, rsp _{total} (arb.)');
\end{lstlisting}
\textit{Plot 3: Total spontaneous emission vs carrier density.}

%==============================================================================
\section{Numerical Method: Peak Finding and Integration}
%==============================================================================

\subsection{Peak Gain Determination}

The code uses a simple maximum search:
\begin{equation}
g_{\max} = \max_{E} \{g(E)\}
\end{equation}

A more sophisticated approach would use gradient-based search or interpolation to find the true maximum between grid points.

\subsection{Trapezoidal Integration for $R_{sp}$}

The spontaneous emission is integrated using a Riemann sum:
\begin{equation}
R_{sp} \approx \sum_{j=1}^{500} r_{sp}(E_j) \cdot \Delta E
\end{equation}

This is equivalent to the trapezoidal rule for uniformly spaced points.

%==============================================================================
\section{Physical Interpretation}
%==============================================================================

\subsection{$\Delta\mu$ vs Carrier Density}

\begin{itemize}
\item $\Delta\mu$ increases with $n$
\item At low $n$: $\Delta\mu < 0$ (absorption)
\item At transparency: $\Delta\mu = 0$
\item At high $n$: $\Delta\mu > 0$ (gain)
\item Relationship is approximately logarithmic in $n$
\end{itemize}

\subsection{Peak Gain vs Carrier Density}

\begin{itemize}
\item Below transparency: $g_{\max} < 0$ (net absorption)
\item Above transparency: $g_{\max}$ increases approximately linearly with $n - n_0$
\item Typical values: $g_{\max} \sim 100$--1000 cm$^{-1}$ for laser operation
\end{itemize}

\subsection{Total Spontaneous Emission}

\begin{itemize}
\item Increases faster than linearly with $n$ (due to Fermi function overlap)
\item Represents fundamental loss mechanism in lasers
\item Determines below-threshold LED emission
\end{itemize}

%==============================================================================
\section{Design Implications}
%==============================================================================

\subsection{Threshold Condition}

At laser threshold:
\begin{equation}
g_{\max} = \alpha_i + \alpha_m
\end{equation}
where $\alpha_i$ is internal loss and $\alpha_m$ is mirror loss.

From the $g_{\max}(n)$ curve, one can determine the threshold carrier density $n_{th}$.

\subsection{Threshold Current}

\begin{equation}
J_{th} = \frac{en_{th}d}{\tau_{sp}(n_{th})}
\end{equation}
where $d$ is the active layer thickness.

%==============================================================================
\section{Summary}
%==============================================================================

This code generates the key curves for semiconductor laser design:
\begin{enumerate}
\item $\Delta\mu(n)$: Determines transparency condition
\item $g_{\max}(n)$: Determines threshold carrier density
\item $R_{sp}(n)$: Determines below-threshold emission and carrier lifetime
\end{enumerate}

These relationships are essential for optimizing laser structures and predicting device performance.

\end{document}
